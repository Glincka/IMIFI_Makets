\section{Оформление текста и рубрикация}

В шаблоне предусмотрено разбиение текста на разделы и подразделы, что реализуется командами \verb|\chapter{}| и \verb|\section{}|. Нумерация формул, определений, рисунков и подобных элементов выполнена согласно номеру раздела и номеру самого элемента, при этом переопределение значений счетчиков не требуется. Для оформления приложений добавлена команда \verb|\appChapter{}|, нумерация всех объектов внутри приложений начинается с буквы данного приложения.

\subsection{Окружения теорем}
Для удобства пользователя определены следующие окружения, которые номеруются в рамках текущей главы:
\begin{itemize}
  \item \verb|Theorem|
  \item \verb|Lemma|
  \item \verb|Example|
  \item \verb|Preposition|
  \item \verb|Remark|
  \item \verb|Proof|
  \item \verb|Corollary|
  \item \verb|defin|
\end{itemize}

\begin{Theorem}
  Алгоритм выполним за время \dots
\end{Theorem}
\begin{Proof}
  Рассмотрим шаги алгоритма \dots
\end{Proof}
\begin{Lemma}
  Если выполняется \dots, то \dots
\end{Lemma}
\begin{Example}
  Рассмотрим следующее выражение \dots
\end{Example}
\begin{Preposition}
  Здесь и далее полагаем \dots
\end{Preposition}
\begin{Remark}
  Важно отметить, что \dots
\end{Remark}
\begin{Corollary}
  Из \dots, следует \dots
\end{Corollary}
\begin{defin}
  Следующее выражение будем называть \dots
\end{defin}

Рисунки вставляются с помощью стандартного окружения \verb|figure|. Например, рис.~\ref{fig1}.

\begin{figure}
\centering
\begin{tikzpicture}
	\node[draw,circle] (a) at (0,0) {$a$};
	\node[draw,circle] (b) at (2,2) {$b$};
	\draw[-Latex,red,thick] (a)--(b) node[midway,sloped,above] {$a\rightarrow b$};
\end{tikzpicture}
\caption{Пример рисунка}
\label{fig1}
\end{figure}

Таблицы вставляются с помощью окружения \verb|table|. Например, таблица~\ref{tab1}.

\begin{table}%
\caption{Пример таблицы}
\centering
\begin{tabular}{|l|c|r|}
\hline
1 & 2 & 3\\\hline
4 & 5 & 6\\\hline
\end{tabular}
\label{tab1}
\end{table}