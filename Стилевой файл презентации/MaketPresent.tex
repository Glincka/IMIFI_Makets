\documentclass[12pt, aspectratio=1610]{beamer}


\usepackage{beamerSFU-VKR}

\institute[ИМиФИ СФУ]{ФГАОУ ВО <<Сибирский федеральный университет>>\\
    Институт математики  и фундаментальной информатики\\
	Кафедра высшей и прикладной математики  \\                                  
}
\direction{Направление 01.04.02 Прикладная математика и информатика\\
Магистерская программа 01.04.02.01 Математическое моделирование}
\supervisor{д.ф.-м.н., профессор, Учёнов У.У.}
\reviewer{к.ф.-м.н., доцент, Строгов С.С.}
\title[Решение очень важной задачи]{Решение очень важной задачи}
\author[Умнов В.В.]{Умнов Василий Васильевич}
	


\date[\today]{Красноярск\\\the\year{}}

\renewcommand*{\thefootnote}{\arabic{footnote}}
\setcounter{footnote}{0}

\begin{document}

\begin{frame}[t, plain]
\titlepage
\end{frame}

\begin{frame}
\frametitle{Актуальность темы исследования}
\begin{block}{}
	Тема исследования является \dots
\end{block}
Основные области применения:
\begin{itemize}
	\item Приложение 1
	\item Приложение 2
	\item \dots
\end{itemize}
\end{frame}

\begin{frame}
\frametitle{Цель и задачи}
\begin{block}{Цель}
Цель исследования в существительной форме (разработка, исследование, анализ, \dots)
\end{block}
\begin{block}{Задачи}
	\begin{enumerate}
		\item Исследовать\dots
		\item Доказать\dots
		\item Разработать\dots
		\item Провести\dots
	\end{enumerate}
\end{block}
\end{frame}

\begin{frame}
	\justifying
	\frametitle{Основные определения и обозначения}
	Здесь и далее даются основные определения, необходимые для постановки математической задачи.

	Можно вводить их в виде текста, а общеизвестные определения ввести в виде таблицы, например:

	\begin{table}
		\centering
		\caption{Основные определения}
		\begin{tabular}{|c|p{.7\textwidth}|}
			\hline
			$V$ & множество вершин\\\hline
			$E$ & множество ребер, определяемое как $E\subseteq V\times V$\\\hline
			$G=(V,E)$ & граф со множеством вершин $V$ и множеством ребер $E$\\\hline
			$s$ & стартовая вершина\\\hline
			$t$ & целевая вершина\\\hline
		\end{tabular}
	\end{table}
	
	\end{frame}

\begin{frame}
		\justifying
		\frametitle{Постановка задачи}
		\begin{block}{Трудная математическая задача}
		\begin{tabular}{ll}
		\textbf{Дано:} & граф $G=(V,E)$, стартовая и целевая вершины $s$ и $t$.\\
		\textbf{Найти:} & кратчайший путь $(s,t)$ и последовательность вершин его образующих.
		\end{tabular}
		\end{block}
		\begin{equation*}
			\lim\limits_{x\rightarrow\infty}{\frac{\sin x}{x}},\quad \int\limits_a^b{x^2\;dx}.
		\end{equation*}
		Здесь можно дать краткую характеристику задачи, например: класс сложности задачи, особенности и пр.
		Можно использовать сноску для вставки литературы\footnotemark.

		\footnotetext{Ученый, А.Б. Очень важные результаты~// А.Б. Ученый~/ Знаменитый математический журнал. -- \No 1(2). -- 2024. -- С.~100-120.}
\end{frame}	

\begin{frame}
	\justifying
	\frametitle{Основные результаты}
	Здесь и далее приводятся основные результаты работы: теоремы, доказательства, алгоритмы и пр.
	\begin{block}{Теорема 1}
		Формулировка всякой теоремы содержит три части:
		\begin{enumerate}
			\item Разъяснительная часть — множество, на котором рассматривается теорема;
			\item Условие — посылка в теореме, сформулированной в условной форме: то есть те положения, при которых заключение имеет место;
			\item Требование (или заключение) — что собственно необходимо доказать, или что об объекте утверждается.
		\end{enumerate}		
	\end{block}
\end{frame}


\begin{frame}[fragile]
	\justifying
	\frametitle{Основные результаты}
	Учтите, что в текущей конфигурации \verb|beamer| позволяет использовать для псевдокода только окружение \verb|algorithmic|.
	\begin{algorithmic}[1]%[5]	
	\scriptsize
	\Require{массив $A[n]$, содержащий $n$ элементов}
	\Ensure{отсортированный массив $A[n]$}
	\For{$j=1$ до $n-1$ с шагом 1}
		\State $f=0$
		\For{$i=0$ до $n-1-j$ с шагом 1}
			\If{$A[i]>A[i+1]$}
				\State Обменять $A[i]$, $A[i+1]$
				\State $f=1$
			\EndIf
			\If{$F=0$}
				\State Выйти из цикла
			\EndIf
		\EndFor
	\EndFor
	\end{algorithmic}
	
\end{frame}

\begin{frame}[fragile]
	\frametitle{Основные результаты}
	В презентации допускается использование двух и более колонок, путем использования окружения \verb!columns!. Выравнивание по высоте задается аргументом \verb|[T], [c], [b]|.
	\begin{columns}[T]
		\column{0.45\textwidth}
		\begin{itemize}
			\item Результат 1
			\item Результат 2
			\item \dots
		\end{itemize}		
		\column{.45\textwidth}
		\begin{figure}
			\includegraphics[width=\textwidth]{IMLogo.png}	
			\caption{Пример рисунка}
		\end{figure}		
	\end{columns}
\end{frame}


\begin{frame}
	\frametitle{Вычислительные эксперименты}
	\justifying
	\label{Exp1}
	\begin{block}{}
		Вычислительные эксперименты должны содержать цель, входные данные, описание и результат.
	\end{block}
	\textbf{Цель:} сравнить алгоритм Alg1 для решения задачи Problem с алгоритмом Alg2.
	
	\textbf{Входные данные:} перечень входных данных.

	\textbf{Описание.} Сравнение алгоритмом осуществлялось по \dots

	\textbf{Результаты.} Приводятся графики, таблицы и иные способы обнародования результатов проведённых экспериментов.	
	
	\hyperlink{Results1}{\beamerbutton{Больше результатов}}
\end{frame}

\begin{frame}
	\frametitle{Заключение}
	\justifying
	Здесь кратко излагаются основные результаты работы, достижение поставленных целей и задач, дополнительные выводы.
	\begin{itemize}
		\item Вывод 1
		\item Вывод 2
		\item Задача 1 решена, получено \dots
		\item \dots
	\end{itemize}
\end{frame}

\begin{frame}
	\frametitle{Аппробация работы}
	\justifying
	Здесь перечисляются все конференции, на которых представлялась работа, и публикации по теме работы.
	\begin{itemize}
		\item XXIII Международная конференция имени А. Ф. Терпугова <<ИНФОРМАЦИОННЫЕ ТЕХНОЛОГИИ И МАТЕМАТИЧЕСКОЕ МОДЕЛИРОВАНИЕ>> (ИТММ -- 2024)
		\item \dots
	\end{itemize}
\end{frame}

\begin{frame}
	\frametitle{Основная литература}
	Здесь приводится самая главная литература по исследованию, на которую ссылались.
	\begin{thebibliography}{9}
		\setbeamertemplate{bibliography item}[article]
		\bibitem{A}
		Ученов, А.Б. Самое важное открытие 2024~// А.Б. Ученов~/ Естетственно научный журнал. -- \No 1(2). -- 2024. -- С.~123-145.
		\bibitem{B}
		Мастеров, В.Г. О новой задаче тысячелетия~// В.Г. Мастеров~/ Математичекий вестник им. Тьюринга. -- \No 4(20). -- 2024. -- С.~1337-1350.
	\end{thebibliography}
\end{frame}

\begin{frame}[plain]
	\centering\Huge Благодарю за внимание!
\end{frame}

\begin{frame}[plain,noframenumbering,fragile]
	\frametitle{Вспомогательные слайды}	
	\justifying
	\label{Results1}
	Вспомогательные слайды следует помещать в конце презентации, после заключительного слова. Их не следует включать в общую номерацию слайдов, что достигается аргументом \verb|noframenumbering|.
	
	Для простоты навигации, можно создавать гиперссылки на слайды командами\linebreak \verb|\hyperlink| и \verb|\label|, например
	
	\hyperlink{Exp1}{\beamerbutton{Вернуться}}

\end{frame}
%%%%%%%%%%%%%%%%%%%%%%%%%%%%%%%%%%%%%%

%%%%%%%%%%%%%%%%%%%
\end{document}
